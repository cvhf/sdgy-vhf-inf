\documentclass[loesung]{schulein}
%
\input{"/Volumes/Chris Nifty/01 Schule/01 Unterrichtsmaterial/TEMPLATES/ab_preamble"}
%
\usepackage{floatflt}
\usetikzlibrary{decorations.pathreplacing}

\kopfDatum{\today} 
\fach{in-z2}
\dokName{Array (Feld)}
\keineSeitenzahlen
%
\begin{document} 
\section*{Datenstruktur Array (Feld)}

\begin{floatingfigure}[r]{3.2cm}
%\vspace*{-.4cm}
\includegraphics[width=3cm]{schachtel-array}
\end{floatingfigure}
%\textbf{Anschauliche Erklärung:} 
\textit{Arrays} (dt.: \textit{Felder}) sind spezielle Variablen, die mehr als einen Wert speichern können. Ein Array kann man sich wie eine Variable als Schachtel vorstellen. Allerdings hat diese Schachtel - im Gegensatz zu der einer einfachen Variablen - durchnummerierte Unterteilungen, in denen die Werte der einzelnen Elemente gespeichert werden.

Auf technischer Ebene ist ein Array eine Aneinanderreihung von Elementen eines festen Datentyps ($T$) im Speicher. Ein Zugriff auf die einzelnen Elemente wird über einen Index ($i$) möglich (vgl. Abb. \ref{fig_1}). Die Nummerierung beginnt mit 0.


%Die Adresse der $i$-ten Komponente lässt sich wie folgt berechnen
%\begin{align*}
%a_0+(i-1)\cdot size(T_0)
%\end{align*}

%\subsubsection*{Technische Umsetzung}
%Arrays werden im Speicher als Aneinanderreihung der Repräsentationen des Grundtyps dargestellt. Auf diese Weise entsteht für ein Array des Typs $T$ zum Grundtyp $T_0$, dessen erstes Element an der Adresse $a_0$ abgelegt wird, die in Abbildung 1 gezeigte Speicherrepräsentation, wobei $R(T_0)$ die Repräsentation eines Elementes des Grunddatentyps ist und $R(T)$ die des gesamten Arrays.

\begin{figure}[h]
\centering
\includegraphics[scale=.8]{figure_1}
\caption{Speicherrepräsentation von Arrays}
\label{fig_1}
\end{figure}

\subsection*{Eigenschaften von Arrays}
\begin{smallitemize}
\item Arrays haben eine feste Größe (Anzahl von Elementen). Sie ist Teil der Deklaration und kann im Anschluss nicht mehr verändert werden.
\item Die Elemente eines Arrays müssen alle denselben Datentyp aufweisen.
\end{smallitemize}


\subsection*{Arrays in Python}
Python stellt von Hause aus keine Implementierung von Arrays zur Verfügung. Sie können aber durch die viel mächtigere eingebaute Datenstruktur \textit{List} (dt.: Listen) realisiert werden. Listen implementieren noch viele Zusatzfunktionen, auf die wir hier aber nicht näher eingehen.

\Ueberschrift{Arbeitsauftrag}{task}
\begin{aufgaben}
\item \leicht Erklären Sie in maximal zwei Sätzen, was ein Array ist.
\item \mittel Erklären Sie mit Hilfe der Abbildung \ref{fig_1} wie die Werte eines Arrays im Speicher organisiert sind. Erläutern Sie zusätzlich die Formel $a_0+i\cdot size(T_0)$. Was lässt sich so berechnen? 
\item \schwer Begründen Sie, warum Arrays nur Elemente des gleichen Datentypen enthalten. %Schreiben sie einen kurzen Text.
\end{aufgaben}



\end{document}